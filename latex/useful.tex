% word count: command line
% https://tex.stackexchange.com/questions/534/is-there-any-way-to-do-a-correct-word-count-of-a-latex-document
\$ detex file.tex | wc -w

%%%%%%%%
% basic document preamble

%% EITHER: Tufte handout
\documentclass{tufte-handout}

%% OR: normal article
\documentclass[11pt]{article}
\special{papersize=210mm,297mm}
\usepackage[top=2.5cm, bottom=2.5cm, left=2.5cm, right=2.5cm]{geometry}  % set margins
\usepackage[usenames, dvipsnames]{xcolor}  % colour package (add to document global options if TikZ in use -- loads xcolor and causes an error: http://tex.stackexchange.com/questions/99049/latex-error-option-clash-for-package-xcolor-even-if-i-put-listings-before)
\usepackage[round]{natbib}  % for citations with round brackets

% no bib, full citations in text
\usepackage{bibentry}  % \bibentry{foo}
\nobibliography*

% useful packages
\usepackage{fourier}  % font
\usepackage[T1]{fontenc}
\usepackage{pifont}  % dingbats for bullets
\usepackage{graphicx}  % insert files
\graphicspath{{images/}}  % update graphics path search
\usepackage{float}  % to position figures precisely using H
\usepackage[nomarkers,figuresonly]{endfloat}  % position figures at end of paper
\usepackage[english]{babel}
\usepackage[nodayofweek]{datetime}  % British date format
\usepackage{hyperref}  % link creation
%\usepackage[colorlinks=true,linkcolor=blue,urlcolor=NavyBlue,citecolor=Fuchsia]{hyperref}
\urlstyle{same} % to keep fonts the same for links
\usepackage{amsmath}  % extended mathematics
\usepackage{booktabs} % book-quality tables, incl 'merge cells' (see below)
\usepackage{units}    % non-stacked fractions and better unit spacing
\usepackage{multicol} % multiple column layout facilities
\usepackage{lipsum}   % filler text
\usepackage{fancyvrb} % extended verbatim environments
  \fvset{fontsize=\normalsize}% default font size for fancy-verbatim environments
\usepackage{multicol}  % for 2 column layout
\setlength{\columnsep}{8mm}
\usepackage{multirow} % for tables
\usepackage{authblk}  % for authors/affiliations
\usepackage{longtable} % for the appendix
\usepackage{arydshln} % for dashed lines in tabular
\usepackage{lscape} % for landscape pages
\usepackage{rotating}  % for sideways figures
\usepackage{enumerate} % for custom numbered lists
\usepackage[normalem]{ulem} % underline without taking over emph command (then \uline{...})
\usepackage{parskip}
%\parskip=0.4\baselineskip \advance\parskip by 0pt plus 1pt
\usepackage{pdfpages}  % include pdf files
\usepackage{fixltx2e}  % for subscript text

\usepackage{fixltx2e}  % provides \textsubscript
\usepackage{array}  % for fixed table width
\newcolumntype{L}[1]{>{\raggedright\let\newline\\\arraybackslash\hspace{0pt}}m{#1}}
\newcolumntype{R}[1]{>{\raggedleft\let\newline\\\arraybackslash\hspace{0pt}}m{#1}}
\usepackage{graphicx}
\newcommand*{\cdotLge}{\raisebox{-0.2ex}{\scalebox{1.6}{$\cdot$}}}  % adjust cdot (bullet) size

\usepackage{geometry}
\geometry{margin=2.5cm}  % set margins

\usepackage{booktabs}
\newcommand{\ra}[1]{\renewcommand{\arraystretch}{#1}}  % then specify line spacing required in table env, e.g.
\begin{table}\centering
\ra{1.3}
\begin{tabular}{ll}\toprule
\begin{tabular}{ | p{2.5cm} | p{1.2cm} | p{3.2cm} | p{3.2cm} | p{3.2cm} | p{1.2cm} | }  % or with paragraph widths specified (m{} or b{} for mid/bottom vertical alignment)
% change text size for whole table (after \begin{table})
{\footnotesize
  \begin{tabular}
  ...
  \end{tabular}
}

\usepackage{scrextend}
\changefontsizes[20pt]{12pt}  % change line spacing and font size

\usepackage[gen]{eurosym}  % for \euro{} € symbol
% n.b. \textdollar and \textcent for $ and ¢ symbols, \pounds for £



%% page header and footer
\usepackage{fancyhdr}
\pagestyle{fancy}
\fancyhf{}
\fancyhead[FLE,FRO]{\thepage} % page numbering
\lhead{\textsc{ALTA: error analysis}}
\chead{}
\rhead{Buttery, Caines, Graham \& McCarthy}

% or foot of page numbering
\pagenumbering{arabic}
\rfoot{Page \thepage \hspace{1pt} of \pageref{LastPage}}

%% section header formatting (here: Cambridge Blue)
\usepackage{titlesec}
\titleformat{\section}
{\color{BlueGreen}\normalfont\large\bfseries}
{\color{BlueGreen}\thesection}{1em}{} % text formatting
\titleformat{\subsection}
{\color{BlueGreen}\normalfont\normalsize}
{\color{BlueGreen}\thesubsection}{1em}{} % text formatting
\titlespacing*{\section}
{0pt}{24pt}{3pt} % section spacing (left edge, above, below)
\titlespacing*{\subsection}
{0pt}{12pt}{3pt} % section spacing (left edge, above, below)

%% for background shading of `code' lines
%% from: http://tex.stackexchange.com/questions/53941/verbatim-environment-with-background-color-pdflatex-and-tex4ht
\usepackage{listings,color}
\definecolor{verbgray}{gray}{0.9}
\lstnewenvironment{code}{%
  \lstset{backgroundcolor=\color{verbgray},
  frame=single,
  framerule=0pt,
  basicstyle=\ttfamily,
  columns=fullflexible}}{}
% usage
\begin{code}
blah
\end{code}

%% define variable in preamble
\newcommand{\user}{apcaines}  % laptop
% and to call it in text just insert:
\user

%% define title and author block
\title{\color{BlueGreen}{Clarity, acceptability, appropriateness: a multi-dimensional framework for error analysis}}
\author{Paula Buttery, Andrew Caines, Calbert Graham, and Michael McCarthy}
\affil{Institute for Automated Language Teaching \& Assessment (ALTA) \\ Department of Theoretical \& Applied Linguistics \\ University of Cambridge}


%%%%%%%%
%% BEGIN
\begin{document}
\maketitle


% custom enumerate
\begin{enumerate}[(A)]
\item as-is,
\item less-disfluency, 
\item less-form-error, 
\item less-all-error.
\end{enumerate}


% change list indents
\usepackage{enumitem}
\begin{itemize}[leftmargin=*]
\item blah
\end{itemize}

% set counter in enumerate lists
\begin{enumerate}
\setcounter{enumi}{1}  % starts counter at #2

% adjust item spacing in lists
\begin{itemize}
\itemsep0em
\item ...

% compact list
\usepackage{paralist}  % compact itemize
\begin{compactitem}
\item ...
\end{compactitem}

% change bullet point style in preamble
% https://texblog.org/2008/10/16/lists-enumerate-itemize-description-and-how-to-change-them/
\renewcommand{\labelitemii}{$\bullet$}  % change 2nd level itemize from dash to bullet


% tables with booktabs
\begin{tabular}{lllr} 
\renewcommand{\arraystretch}{1.5}  % adjust spacing between rows
\toprule
& \multicolumn{2}{c}{Name} & Grade \\ 
\midrule
\multirow{2}{*}{x} & John & Doe & $7.5$ \\  # multirow for n rows, * indicates width (value or natural spacing), x is text
& Richard & Miles & $2$ \\ 
\bottomrule 
\caption{\label{results} Mean parse probabilities }
\end{tabular}


% include graphics as part of figure
% reqs \usepackage{graphicx} and \usepackage{float} for position H
\begin{figure}[H]
\begin{center}
\includegraphics[width=\linewidth]{clc-2009_pos-per-100-words_V.pdf}
\end{center}
\caption{\label{refLabelHere} \small pithy caption here}
\end{figure}
% turn a Figure to landscape
% reqs \usepackage{lscape}
\begin{landscape}
% figure stuff as above...
\end{landscape}

% sideways figures
\begin{sidewaysfigure}
\centering
\includegraphics[scale=0.6]{screenshots/whole-map} 
\caption{The caption of the figure.}
\label{fig:map}
\end{sidewaysfigure}


% special characters
$\mu$  % mu
$\delta$  % delta
$\neg$  % ¬ negation
$\langle$ $\rangle$  % left and right angled brackets
% or in text mode use textcomp package
\usepackage{textcomp}
\textlangle \textrangle
\usepackage{textcomp,    % for \textlangle and \textrangle macros
            xspace}
\newcommand\la{\textlangle\xspace}  % set up short-form macros with following space
\newcommand\ra{\textrangle\xspace}

\greektext a \latintext  % alpha
% package 
\usepackage{textgreek}
\textomega  % see http://texblog.org/2012/03/15/greek-letters-in-text-without-changing-to-math-mode/
\textsigma
\textmu

\textsuperscript{x}
\textsubscript{x}  % requires \usepackage{fixltx2e} in preamble


% for ACL bibliographies, to ensure casing preserved in journal titles etc
% in acl.bst file, change:
{ title ``t'' change.case\$ }
% for:
{ title }


% include PDF file, reqs \usepackage{pdfpages}
\includepdf[pages={1}]{myfile.pdf}
\includepdf[pages={1,3,5}]{myfile.pdf}
\includepdf[pages={-}]{myfile.pdf}  % all pages
\includepdfmerge[landscape, nup=1x2]{Admin/AGMs/summary-NTS-CTS-survey_JoMc_2014-09-30.pdf, -, Admin/AGMs/leagues-report_JoMc_2014-09-30.pdf, -}  % list of PDFs, landscape, 1 row 2 col layout (i.e. pages side-by-side)


% making author / affiliation blocks, though note post about multiple author affiliations: http://josef.spjut.me/latex/2014/01/03/multiple-author-affiliations-in-latex/
\usepackage{authblk}  % for authors/affiliations
\author[1]{author1}  % footnote style
\author[1]{author2}
\author[2]{author3}
\affil[1]{affil1}
\affil[2]{affil2}


% define page margins
\usepackage[top=2.5cm, bottom=2.5cm, left=2.5cm, right=2.5cm]{geometry}  % set margins


% multi column layout
\usepackage{multicol}
\setlength{\columnsep}{8mm}
\begin{multicols}{2}
\end{multicols}


% no 'abstract' header
\renewcommand{\abstractname}{\vspace{-\baselineskip}}


% insert today's date
\today


% general switch to biblatex bibliographies
% see http://tex.stackexchange.com/questions/13509/biblatex-for-idiots
% and https://www.sharelatex.com/blog/2013/07/31/getting-started-with-biblatex.html
\documentclass{article}
\usepackage[autocite=footnote,backend=bibtex,style=authoryear,sorting=nyt,maxbibnames=99]{biblatex}
\addbibresource{<database>.bib}
\begin{document}
\cite{<some-ref>}
\printbibliography
\end{document}
% note 8-bit bibtex command
\$ bibtex8 --wolfgang <filename>


% tabbing: http://tex.stackexchange.com/questions/15866/setting-tabs-by-declaration-rather-than-by-example
\documentclass{article}
\begin{document}
\begin{tabbing}
\hspace*{2cm}\=\hspace*{3cm}\= \kill
column1a \> column2a \> column3a \\
column1b \> column2b \> column3b 
\end{tabbing}
\end{document}


% long underscore
\underline{\hspace{3cm}}


% for Chinese/Japanese characters
% preamble
\usepackage[koi8-r,utf8,latin1]{inputenc}  % amended LREC line (commented out above)
\usepackage{CJKutf8} 
\usepackage{ucs} 
\usepackage[encapsulated]{CJK}
% after begin doc
\begin{CJK*}{UTF8}{min}
% end of doc
\end{CJK*}

% or use 'xelatex' to compile
\usepackage{xeCJK}
\setCJKmainfont[BoldFont=FandolSong-Bold.otf]{FandolSong-Regular.otf}
\setCJKsansfont[BoldFont=FandolHei-Bold.otf]{FandolHei-Regular.otf}
\setCJKmonofont{FandolFang-Regular.otf}
\newCJKfontfamily\kaiti{FandolKai-Regular.otf}



% caret in text mode
\string^

% tilde
\textasciitilde
% maths mode
\sim

% less than / greater than
\usepackage[T1]{fontenc}
\textless
\textgreater


% enclose text in boxes
\framebox{foo}


% longtables
\usepackage{longtable} % for the appendix
\usepackage{tabu}
\usepackage{lscape}  % rotate page

% if two column article clear page
\newpage
\onecolumn

\begin{center}
%\begin{longtable}{|l|l|l|l|}
\begin{longtabu} to \textwidth {|X[0.4 , p ] |X[0.3, l ] | X[0.9 , l ]| X[1 , l ]|}\firsthline\hline
\caption[List of features extracted from Weibo posts]{List of features extracted from Weibo posts.} \label{tbl:feats} \\

\hline
\textbf{Feature} & \textbf{Type} & \textbf{Description} & \textbf{Example} \\
\hline
\hline 
\endfirsthead

\multicolumn{3}{c}%
{{\bfseries \tablename\ \thetable{} -- continued from previous page}} \\
\hline
\textbf{Feature} & \textbf{Type} & \textbf{Description} & \textbf{Example} \\
\hline
\hline 
\endhead

\hline \multicolumn{3}{|r|}{{Continued on next page}} \\ \hline
\endfoot

\hline
\endlastfoot

content & . & . & . \\

\end{longtabu}
\end{center}


% adjust paragraph indentation
\usepackage{lipsum}       % for sample text
\usepackage{changepage}   % for the adjustwidth environment
\begin{document}
\lipsum[1]
\begin{adjustwidth}{2cm}{}
\lipsum[1]
\end{adjustwidth}
% Following Werner's comment, you could make your environment take an optional argument that would overwrite the default indentation:
\newenvironment{myenv}[1][2cm]{\begin{adjustwidth}{#1}{}}{\end{adjustwidth}}
\begin{myenv}
\lipsum[1]
\end{myenv}


% insert a dagger symbol, i.e. †
$\dagger$

% pipe symbol
\usepackage[T1]{fontenc}
then type pipe as normal |

% normal spacing after full stop
Prompts are all of the form, \emph{Frag:...}\ (`ask')


% IPA symbols: https://www.tug.org/tugboat/tb17-2/tb51rei.pdf
\usepackage{tipa}
\textipa{[TIsIzs@maIpieI]}
\textipa{[Its\*rilijizitutaIp]}


% reference a page number
\pageref{foo}


% split/align an equation
\usepackage{amsmath}
% split (unnumbered)
\begin{multline*}
p(x) = 3x^6 + 14x^5y + 590x^4y^2 + 19x^3y^3\\ 
- 12x^2y^4 - 12xy^5 + 2y^6 - a^3b^3
\end{multline*}
% align
\begin{align*} 
2x - 5y &=  8 \\ 
3x + 9y &=  -12
\end{align*}
